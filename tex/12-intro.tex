\Introduction
Реляционная модель данных нужна для организации данных в виде таблиц (отношений) с помощью связей - полей таблиц, 
ссылающихся на другие таблицы, которые называются внешними ключами.  
Сегодня существуют другие модели данных, включая NoSQL, но системы управления реляционными базами данных (СУБД) остаются доминирующими для хранения 
и управления данными во всем мире \cite{db-engines-rating}. 

Одними из самых популярных СУБД являются MySQL и PostgreSQL. 

MySQL --- свободная реляционная система управления базами данных. 

PostgreSQL - свободная объектно-реляционная система управления базами данных.
Postgres более функциональная СУБД, нежели MySQL хотя и сложнее для использования.


Часто, с развитием проекта, возникает необходимость миграции с одного движка СУБД на другой. 
Одной из причин является нехватка функциональных возможностей.  

PostgreSQL является популярным выбором для функций NoSQL. 
Она изначально поддерживает большое разнообразие типов данных, включая JSON, hstore и XML \cite{postgres-types}.

Управление параллельным доступом посредством многоверсионности (MVCC) -  это одна из главных причин, почему компании выбирают PostgreSQL. 
MVCC предоставляет одновременный доступ к базе данных множеству агентов на чтение и запись. 
Это устраняет необходимость каждый раз блокировать чтение-запись, когда кто-то взаимодействует с данными. 
Таким образом, значительно повышается эффективность управления СУБД и ее производительность.

MVCC обеспечивает такую функциональность через изоляцию снапшотов. 
Моментальные снимки (снапшоты) представляют состояние данных в определенный момент времени.

Цель работы --- исследовать совместимость диалектов sql для СУБД MySQL и PostgreSQL для решения задачи по конвертации запросов.
Таким образом, необходимо выполнить следующие задачи:
\begin{itemize}
\item выяснить совместимость СУБД с стандартом SQL;
\item выявить различия диалектов PostgreSQL, MySQL;
\item определить взаимозаменяемость конструкций;
\item описать существующие решения по конвертации запросов MySQL в PostgreSQL;
\item выяснить особенности и их преимущества;
\item сформулировать выводы.
\end{itemize}

\Abbrev{СУБД}{Система управления базами данных "" --- совокупность программных средств, обеспечивающих управление созданием и использованием баз данных}
\Abbrev{SQL}{Structured Query Language ""--- язык структурированных запросов}
\Abbrev{NoSQL}{Not only Structed Query Language""--- термин, обозначающий ряд подходов, направленных на реализацию хранилищ баз данных, имеющих существенные отличия от моделей, используемых в традиционных реляционных СУБД с доступом к данным средствами языка SQL}
\Abbrev{MVCC}{Multiversion Concurrency Control ""--- один из механизмов СУБД для обеспечения параллельного доступа к базам данных, заключающийся в предоставлении каждому пользователю так называемого «снимка» базы, обладающего тем свойством, что вносимые пользователем изменения невидимы другим пользователям до момента фиксации транзакции}
\Abbrev{JSON}{JavaScript Object Notation ""--- текстовый формат обмена данными, основанный на JavaScript}
\Abbrev{XML}{Extensible Markup Language ""--- расширяемый язык разметки, используемый для хранения и передачи данных}
\Abbrev{ANSI}{American National Standards Institute ""--- объединение американских промышленных и деловых групп, разрабатывающее торговые и коммуникационные стандарты}
\Abbrev{ISO}{International Organization for Standardization ""--- международная организация, занимающаяся выпуском стандартов}



\Define{PostgreSQL}{свободная объектно-реляционная система управления базами данных}
\Define{MySQL}{свободная реляционная система управления базами данных}
\Define{Microsoft SQL Server}{cистема управления реляционными базами данных, разработанная корпорацией Microsoft}





