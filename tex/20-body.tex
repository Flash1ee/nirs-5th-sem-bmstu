\chapter{Анализ предметной области}
\label{cha:body}
Базы данных --- это упорядоченный набор структурированной информации или данных, которые обычно хранятся в электронном виде в компьютерной системе. 
База данных обычно управляется системой управления базами данных (СУБД). 
Система управления базами данных --- это программа, которая взаимодействует с базой данных \cite{oracle-db}. 
СУБД позволяет контролировать доступ к базе данных, записывать данные, выполнять запросы и любые другие задачи, связанные с управлением базой данных. 

База данных может быть любым набором данных, а не только хранящимся на компьютере, СУБД --- это программное обеспечение, которое позволяет взаимодействовать с базой данных.

Большинство реляционных баз данных используют язык структурированных запросов (SQL) для управления данными и запросов к ним.
Однако многие СУБД используют свой собственный диалект SQL, который может иметь определенные ограничения или расширения. 
Эти расширения обычно включают в себя дополнительные функции, которые позволяют пользователям выполнять более сложные операции, 
чем они могли бы делать, использую стандартный SQL. 

Стандартный SQL --- язык структурированных запросов, поддерживаемый организациями ANSI\cite{ansi-sql} и ISO\cite{iso-sql}. 
В стандарте используется модульная структура, основная часть стандарта вынесена в раздел "SQL/Foundation".  
Для современных СУБД существует требование по соблюдению стандарта --- они должны поддерживать основной модуль. 
Однако полной поддержки стандарта не предоставляет ни одна из систем.  
Так, стандарт SQL 2016 года содержит 177 функций в основном модуле. PostgreSQL реализует 170 из них \cite{postgresql-standard},
MySQL --- чуть более 140 \cite{mysql-standard}.  

По статистике на январь 2022 года \cite{sql-rating}, MySQL занимает 2 место, Microsoft SQL Server 3 место в мире по популярности 
среди реляционных СУБД. PostgreSQL находится на 4 месте. 


Исходя из больших функциональных возможностей и лучшей поддержкой управления параллельным доступом, 
переход с MySQL на PostgreSQL является популярной задачей разработки. 

В научно-исследовательской работе предстоит выяснить, чем отличаются данные СУБД, совместимы ли они, 
как перейти с одной на другую.  

% %%% Local Variables:
% %%% mode: latex
% %%% TeX-master: "rpz"
% %%% End:
